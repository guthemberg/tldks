\section{Related Work}
\label{sec:related_work}

Our related work is two-fold: Internet videos and adaptive replication schemes.

\noindent \textbf{Internet videos:} Recent studies
\cite{youtube_wsdm_2011,popularity_prediction_2010} have drawn
attention to reach a better understanding of Internet videos
properties, such as popularity growth. They point out that well-known
popularity characteristics are applicable to multimedia content. For
instance, Internet videos popularity distribution follows power law,
and popularity bursts have a short duration and are quite likely to
happen just after the content publication. Dobrian \emph{et
al.}~\cite{Dobrian_sigcomm_2011} shed some light on the performance of
Internet videos provision on CDNs. They show that average bitrate plays
an important role in videos availability. A hybrid solution between
CDNs and P2P is presented by Mansy \emph{et
al.}~\cite{Mansy_icnp_2011}. Their purpose is to model and analyze a
live video system and one of their main concerns is to adapt bitrate
for guarantee user satisfaction. Adhikari \emph{et
al.}~\cite{Adhikari_infocom_2012} work described the YouTube video
delivery system through measurements of DNS resolutions and video
playback traces. One of their findings is that over a globally
distributed network (PlanetLab) most part of the nodes have a nearby
Youtube video cache server to delivery the video data.  Moreover,
Brodersen \emph{et al.}~\cite{Brodersen_www_2012} presented a detailed
study over the strong connection between popularity and geographic
locality of Youtube videos. These facts endure our decision of a
locality aware solution for infrastructure. Liu \emph{et
al.}~\cite{Liu_sigcomm_2012} make a case for a video control plane
that can use a global view of client and network conditions to
dynamically optimize the video delivery in order to provide a high
quality viewing experience despite an unreliable delivery
infrastructure. However, the granularity of their server selection
mechanism is at a CDN, ignoring edge network resources. WiseReplica
addresses this issue by adapting replication close to the viewers.
Thus, WiseReplica can be play an important role in collaborating with an
Internet control plane.


\noindent
\textbf{Adaptive replication schemes:} Non-collaborative caching remains the simplest approach to provide popularity-aware replication of web content through cache replacement policies\cite{popularity_awaregreedydual_size_icdcs99}. However, we showed when we adapt the number of replicas according to the Internet video popularity properly, cache replacement policy becomes redundant. EAD \cite{Haiying_Shen_P2P_2010}
and Skute \cite{self_tolerant_acm_cloud_2010} adapt the number of replicas by
using a cost-benefit approach over decentralized and structured P2P
systems. EAD creates and deletes replicas throughout the query path
with regard to object hit rate using an exponential moving average
technique. Similarly, Skute provides a replication management scheme that
evaluates replicas price and revenue across different geographic
locations. Despite presenting an
efficient framework for replication, they provide an inaccurate bitrate provision, hence
inappropriate for high-quality video delivery. WiseReplica copes with this issue through analysing the request arrival process, performing accurate predictions about the ranking of Internet videos, and maintaining replication degree accordingly.
