\section{Introduction}

The increasing consumption of Internet videos has made fundamental changes in the Internet traffic and consumers' behaviour. Cisco System, Inc\footnote{Cisco Visual Networking Index: Forecast and Methodology, 2011-2016. www.cisco.com, 2012.} forecasts that the video traffic will reach 86\% of the global consumer traffic by 2016, including video on-demand (VoD), live streaming, and peer-to-peer (P2P) file sharing. In fact, as the Internet access has become ubiquitous, continuously faster, and cheaper, streaming video has become mainstream. Users are progressively moving from the old-fashioned scheduled television to VoD services. This contributes to increase the expectations of consumers on Internet video delivery. 

Since broadcasting future seems to be online, customers have become more sensitive to VoD quality, expecting ever-higher bitrates and lower rebuffering. Contrary to many traditional workloads, e.g. social network messaging or search engines, specifying just latency as QoS metric does not suffice. Instead, streaming traffic requires proper average bitrate to avoid rebuffering and to improve user experience. For example, Dobrain \emph{et al.}\cite{Dobrian_sigcomm_2011} found that a 1\% increase in buffering ratio can reduce the consumer's expected viewing time by more than three minutes. This suggests that SLA contracts must include bitrate as a key QoS metric. 

Yet current CDN architectures are not ready to fulfil the requirements of the increasing demand for streaming and meet consumers' expectations. Through fine-grained client-side measurements from over 200 million client viewing sessions, Liu \emph{et al.}\cite{Liu_sigcomm_2012} showed that 20\% of these sessions experience a re-buffering ratio of at least 10\%, 14\% of users have to wait more than 10 seconds for video to start up, more than 28\% of sessions have an average bitrate less than 500Kbps, and 10\% of users fail to see any video at all.

To deal with these issues, CDN providers have started to combine datacenters and edge network resources in hybrid designs\footnote{Akamai acquires Red Swoosh. http://www.akamai.com/html/about/ press/releases/ 2007/press\_041207.html, April 2007.}. This includes peer-assisted VoD systems. The aim is to take advantage of both infrastructure-based resources and P2P communication facilities. Huang \emph{et al.}~\cite{profitable_vod_sigcomm_07} suggest peer-assisted VoD systems improves resource allocation for Internet video delivery. They argue that devices on edge networks, e.g. set-top-boxes, contribute with storage and bandwidth to video delivery, reducing dramatically the burden on infrastructure-based servers, and cutting operations costs. Many recent studies~\cite{parvez_bittorrent_analysis_sigmetrics08,huang2008challenges_sigcomm08,pavod_icnp12} confirm that exploring peer-assisted VoD system permits enhancing resource allocation for streaming videos, but none has properly evaluated the performance of video delivery regarding SLA enforcement. 

Actually there exists an increasing need for more research in easy-to-deploy, self-adapting techniques for ensuring tough QoS guarantees brought by the cloud paradigm. However, efficient resource allocation on hybrid CDNs to meet user expectations imposes big challenges, particularly for resource-hungry services as VoD. 
This paper identifies \emph{adaptive content replication} as one of such challenges. Adaptive replication plays an important role on the content availability of distributed systems, contributing directly to both storage and bandwidth provision. As the popularity of a video varies, the number of replicas, or peers serving that video, must be adapted accordingly. Generally speaking, the faster and more precise the replication scheme reacts to changes on videos popularity, the better is the resource allocation. 

Considering average bitrate as target QoS metric, we make a case for a SLA-driven replication scheme named WiseReplica that allows us to meet users' expectations in peer-assisted VoD system properly. We assume the system must enforce the right average bitrate for each video through SLA contracts. Our ultimate goal is two-fold, to prevent SLA violations and to reduce the number of video replicas. For performing efficient replication, WiseReplica relies on a novel, accurate machine-learned ranking of Internet video \emph{hotness}. Our rank-based replication approach permits adapting dynamically the replication degree of videos according to their encoding settings and popularity, reducing storage usage and enhancing network provision.  We make two main contributions:

\noindent
\textbf{Investigate how predictable is a ranking of Internet videos}. We design a learning model to capture the dynamic behaviour of streaming video demand. The model makes predictions based on lightweight measurements of the request arrival process. Using a novel machine-learned ranking, we predict the \emph{hotness} of a video accurately. Thus, the higher the rank position, the higher the demand for fresh replicas. This intuitive model allows us to decouple streaming demand from replication policy. Hence one can evaluate different replication policies regardless of the workload. Our model is flexible, and can learn from different sources and big amounts of data, providing a robust framework for controlling VoD resource allocation. Simulations using YouTube traces, with non-stationary behaviours, suggest that our model is very accurate in predicting video \emph{hotness}. Since our ranking of videos is based on random forests, a parallelizable, state-of-the-art machine learning method, it fits requirements for going online.

\noindent
\textbf{Enforce average bitrate through SLA-based video replication}. We designed and evaluated WiseReplica, an easy-to-deploy, SLA-based replication scheme that meets users' expectations for VoD services. WiseReplica is fully compliant with peer-assisted VoD systems. It operates adaptive replication over sets of devices located close by in edge networks, namely \emph{storage domains}. WiseReplica functioning per storage domain is straightforward. Gradually, it verifies the rank position of a video whenever a new local request arrives, and adapts the replication degree accordingly. Using a collaborative caching, video replicas are either pre-fetched or removed randomly. We show through simulations using YouTube traces that WiseReplica outperforms a non-collaborative caching by preventing violations, reducing storage usage, and enhancing network provision. Furthermore, our replication scheme is easy to adopt and flexible enough to offer interoperability with de facto approaches, including HTTP adaptive streaming technique and BitTorrent protocol\cite{bittorrent_P2P_protocol}.


This work is organized as follows. In Section~\ref{sec:context} 
we present the context and challenges of this research. We describe in details our learning model in Section~\ref{sec:learning_model}. Section~\ref{sec:replication_scheme} describes WiseReplica design and approach. We explain our simulation methodology in Section~\ref{sec:simulation_methodology}. Then we analyse the performance of WiseReplica in Section~\ref{sec:evaluation}. We place our contributions on related works in Section~\ref{sec:related_work}, just before concluding in Section~\ref{sec:conclusion}.
