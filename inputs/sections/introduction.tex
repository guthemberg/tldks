\section{Introduction}

The increasing consumption of Internet videos has made fundamental changes in the Internet traffic and consumers' behaviour. Cisco System, Inc\footnote{Cisco Visual Networking Index: Forecast and Methodology, 2013-2018. www.cisco.com, 2014.} forecasts that the sum of all forms of video traffic will be in the range of 80 to 90 percent of global consumer traffic by 2018, including video on-demand (VoD), live streaming, and peer-to-peer (P2P) file sharing. In fact, as the Internet access has become ubiquitous, continuously faster, and cheaper, streaming video has become mainstream. Users are progressively moving from the old-fashioned scheduled television to VoD services. This contributes to increase the expectations of consumers on Internet video delivery. 

Since broadcasting future seems to be online, customers have become more sensitive to VoD quality, expecting ever-higher bitrates and lower rebuffering. Contrary to many traditional workloads, e.g. social network messaging or search engines, specifying just latency as quality of service (QoS) metric does not suffice. Instead, streaming traffic requires proper average bitrate to avoid rebuffering and improve user experience. For example, Dobrain \emph{et al.}\cite{Dobrian_sigcomm_2011} found that a 1\% increase in buffering ratio can reduce the consumer's expected viewing time by more than three minutes. Balachandran \emph{et al.}, observe that increased average bitrate in Internet video delivery leads to a better user experience for viewers with mobile devices~\cite{balachandran2013developing}.This suggests that service-level agreement (SLA) contracts must include average bitrate as a key QoS metric. 

Yet, current Content Delivery Networks (CDN) platforms are not ready to fulfil the requirements of the increasing demand for VoD services and meet consumers' expectations. Through fine-grained client-side measurements from over 200 million client viewing sessions, Liu \emph{et al.}\cite{Liu_sigcomm_2012} showed that 20\% of these sessions experience a rebuffering ratio of at least 10\%, 14\% of users have to wait more than 10 seconds for video to start up, more than 28\% of sessions have an average bitrate less than 500Kbps, and 10\% of users fail to see any video at all.

To deal with these issues, CDN providers have started to combine
datacenters and edge network resources in hybrid
designs\footnote{Akamai acquires Red
  Swoosh. http://www.akamai.com/html/about/ press/releases/
  2007/press\_041207.html, April 2007.}. This includes
\emph{peer-assisted VoD systems}~\cite{profitable_vod_sigcomm_07}
whose deployment requires hybrid CDN platforms. The aim of
peer-assisted VoD systems is to take advantage of both
infrastructure-based resources and P2P communication facilities. Huang
\emph{et al.}~\cite{profitable_vod_sigcomm_07} suggest the use of
peer-assisted VoD systems to improve resource allocation for Internet video delivery. They argue that devices on edge networks, e.g. set-top-boxes, contribute with storage and bandwidth to video delivery, reducing dramatically the burden on infrastructure-based servers, and cutting operations costs. Many recent studies~\cite{parvez_bittorrent_analysis_sigmetrics08,huang2008challenges_sigcomm08,pavod_icnp12} confirm that exploring peer-assisted VoD system permits enhancing resource allocation for streaming videos, but none has properly evaluated the performance of video delivery regarding SLA enforcement. 

In fact, there exists an increasing need for more research in easy-to-deploy, self-adapting techniques for ensuring tough QoS guarantees brought by the cloud paradigm. However, efficient resource allocation on hybrid CDNs to meet user expectations imposes big challenges, particularly for resource-hungry services as VoD. 
This paper identifies \emph{adaptive content replication} as one of such challenges. Adaptive replication plays an important role on the content availability of distributed systems, contributing directly to both storage and bandwidth provision. As the popularity of a video varies, the number of replicas, or peers serving that video, must be adapted accordingly. Generally speaking, the faster and more precise the replication scheme reacts to changes on videos demand, the better is the resource allocation and content availability. 

Considering average bitrate as target QoS metric, we make a case for a
SLA-driven replication scheme named WiseReplica that allows us to meet
users' expectations in peer-assisted VoD system properly. We assume
the system must enforce the right average bitrate for each video
through SLA contracts. Our ultimate goal is two-fold: (i) to prevent
SLA violations and (ii) to reduce the number of video replicas. To
perform efficient Internet video replication, WiseReplica relies on a
novel, accurate machine-learned ranking of Internet videos. To rank
video in order of demand,  our prediction model encompasses multiple
measurements of Internet video activity in peer-assisted VoD system,
including active viewers, video duration, average serving time, and
mean time between requests view. The use of this prediction model in
WiseReplica provides the ability to adapt the replication degree of videos dynamically according to their encoding settings and popularity, reducing storage usage and enhancing network provision.  We make two main contributions:

\noindent
\textbf{Investigate how predictable is a ranking of Internet videos.} \ We design a learning model to capture the dynamic behaviour of streaming video demand. The model makes predictions based on lightweight measurements of the request arrival process. Using a novel machine-learned ranking, we predict demand of a video accurately. Thus, the higher the rank position, the higher the demand for fresh replicas. According to the video ranking position, VoD services operators can define and evaluate different replication policies. For instance, top-ranked Internet videos may be twice as much replicated as those ranked in the second position. This intuitive model allows us to decouple streaming demand from replication policy. Our model is flexible and can learn from different sources and big amounts of data, providing a robust framework for controlling VoD resource allocation. Simulations using YouTube traces, with non-stationary behaviours, suggest that our model is very accurate in predicting the ranking of Internet videos. Since our ranking of videos is based on random forests, a parallelizable, state-of-the-art machine learning method, it fits runtime requirements of large VoD systems.

\noindent
\textbf{Enforce average bitrate through SLA-based video replication.}
\ Based on our machine-learned ranking of Internet video, we designed
and evaluated WiseReplica, an easy-to-deploy, SLA-based replication
scheme that meets users' expectations for VoD services. WiseReplica is
fully compliant with peer-assisted VoD systems in hybrid CDN
platforms. It operates adaptive replication over sets of devices
located close to each other in edge networks, namely \emph{storage
  domains}. WiseReplica functioning per storage domain is
straightforward. Gradually, it verifies the rank position of a video
whenever a new local request arrives, and adapts the replication
degree accordingly. Using a collaborative caching, video replicas are
either pre-fetched or removed randomly. We show through simulations
using YouTube traces that WiseReplica outperforms a non-collaborative
caching approach by preventing violations, reducing storage usage, and enhancing network resources provision. Furthermore, our replication scheme is easy to adopt and flexible enough to offer interoperability with de facto approaches, including HTTP adaptive streaming technique and BitTorrent protocol~\cite{bittorrent_P2P_protocol}.


This work is organized as follows. In Section~\ref{sec:context}, 
we present the context and challenges of this research. We describe in
details our prediction model to rank Internet videos in order of
demand in
Section~\ref{sec:learning_model}. Section~\ref{sec:replication_scheme}
describes the approach of our adaptive replication scheme,
WiseReplica. We explain our simulation methodology in
Section~\ref{sec:simulation_methodology}. We then analyse the
performance of WiseReplica in Section~\ref{sec:evaluation}. Related
works are discussed in Section~\ref{sec:related_work}, just before the
conclusion in Section~\ref{sec:conclusion}.
